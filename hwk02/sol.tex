\documentclass[11pt]{article}
% Adjust the headheight and topmargin
\setlength{\headheight}{22.54448pt}
\addtolength{\topmargin}{-10.54448pt}

%%%%%%%%%%%%%%%%%%%%%%%%%%%%%%%%%%%%%%%%%%%%%%%%%%%%%%%%%%%%%%%%%%%%%%%%%%%%%%%%%%%%%%%%%%%%%%%%%%%%%%%%%%%%%%%%%%%%%
%																																																										%
%										PAGE SETTINGS and PACKAGE LOADING. (You do not need to edit the scripts below.)									%
%																																																										%
%%%%%%%%%%%%%%%%%%%%%%%%%%%%%%%%%%%%%%%%%%%%%%%%%%%%%%%%%%%%%%%%%%%%%%%%%%%%%%%%%%%%%%%%%%%%%%%%%%%%%%%%%%%%%%%%%%%%%
			% Margin setting
			\usepackage[
				letterpaper,
				top=1.8cm,
				bottom=1.5cm,
				left=1.8cm,
				right=1.8cm,
				includefoot,
				includehead
			]{geometry}

			% List spacing setup
			\setlength{\topsep}{0pt}
			\setlength{\partopsep}{0pt}
			\setlength{\listparindent}{0em}
			\setlength{\labelwidth}{\leftmargin}
			%% set the vertical spacing between paragraphs
			\setlength{\parskip}{1.5mm}

			% Find the last page
			\usepackage{lastpage}

			% Fancy headings
			\usepackage{fancyhdr}
			\pagestyle{fancyplain}

			\newcommand{\hwnumber}[1]{\def\hwnumberdata{#1}}
			\def\hwnumberdata{\relax}

			\renewcommand{\author}[1]{\def\authordata{#1}}
			\def\authordata{\relax}
			\newcommand{\collaborators}[1]{\def\collaboratorsdata{#1}}
			\def\collaboratorsdata{\relax}

			\fancyhead[L]{\small CS 330 Homework \#\hwnumberdata \\ %Problem \#\problemnumberdata \\
				\textsl{Collaborators}: \collaboratorsdata}
			\fancyhead[R]{\small \authordata \\ }
			\fancyfoot[C]{\small \thepage\ of \pageref{LastPage}}

			\RequirePackage{titlesec}
			\titleformat{\subsection}{\normalsize\bfseries}{\thesubsection}{.5em}{}
			\renewcommand{\thesubsection}{\alph{subsection})}

			% For including image files
			\usepackage{graphicx}
			\usepackage[ruled,vlined,noline]{algorithm2e}

			% For fancy math
			\RequirePackage{amsmath,amsthm,amssymb}
			\newtheorem{theorem}{Theorem}
			\newtheorem{fact}[theorem]{Fact}
			\newtheorem{lemma}[theorem]{Lemma}
			\newtheorem{claim}[theorem]{Claim}

			\newcommand{\ord}[2][th]{\ensuremath{{#2}^{\mathrm{#1}}}}
			% shorthand for \mathcal{O}
			\newcommand{\Ocal}{\ensuremath{\mathcal{O}}}
%
%			PAGE SETTING SCRIPTS END HERE		--->
%
%%%%%%%%%%%%%%%%%%%%%%%%%%%%%%%%%%%%%%%%%%%%%%%%%%%%%%%%%%%%%%%%%%%%%%%%%%%%%%%%%%%%%%%%%%%%%%%%%%%%%%%%%%%%%%%%%%%%%




% homework number
\hwnumber{2}

% your name
\author{Josh Mayer}

% Collaborators. If you didn't collaborate, write "\collaborators{none}".
% If you did, for each collaborator, write "worked together", "I helped him/her" or "He/she helped me".
\collaborators{}


\begin{document}
\section{Problem 1}

% If the problem has multiple parts, use \subsection command.
\subsection{True}

Using the limit method and algebra we
can see that the limit evaluates to a constant.

\begin{equation*}
    \lim _{n\to \infty }(\frac{2^{n+1}}{2^n})=
    \lim _{n\to \infty }(2^{(n+1)-(n)})=
    \lim _{n\to \infty }(2^{1})=
    2
\end{equation*}
Therefore, it must be true that $2^{n+1} = O(2^{n})$.

\subsection{False}

Using the limit method and algebra we
can see that the limit evaluates to infinity.

\begin{equation*}
    \lim _{n\to \infty }(\frac{2^{2n}}{2^n})=
    \lim _{n\to \infty }(2^{(2n)-(n)})=
    \lim _{n\to \infty }(2^{n(2-1)})=
    \lim _{n\to \infty }(2^{n})=
    \infty
\end{equation*}
Therefore, it must not be true that $2^{2n} = O(2^{n})$.

\section{Problem 2}
\begin{proof} Direct proof.
Assume for functions $f$, $g$, and $h$ that $f(n) = O(g(n))$ and $g(n) = O(h(n))$.
Therefore it must be true that for constants $c$ and $d$, that $f(n) \le c * g(n)$ and $g(n) \le d * h(n)$.
Substituting in $g(n)$, you get $f(n) \le c(d*h(n))$. This is the same as $f(n) \le (c*d) h(n)$.
Because $c*d$ is a constant we have shown that $f(n) = O(h(n))$ for sufficiently large n.
\end{proof}

\section{Problem 3}
\subsection{Yes}
This is true because the time required to read in the entire array input is $\Omega{(n)}$ and its impossible to make
an algorithm that is faster than being able to see every item in the list.

\subsection{$O(n\log{n})$}
The overall complexity is $O(n\log{n})$ because the two pieces of code run sequentially which means their runtime
is added together. So in this case $(n\log{n}) + n = O(n\log{n})$.

\subsection{$O(n^{2})$}
The two different parts are also run sequentially, so their complexities add together as well. This means the overall
complexity looks like $(n * \log{n}) + (n * n) = (n\log{n}) + (n^{2})$. This time complexity is bounded by the polynomial
which makes the overall complexity $O(n^{2})$.

\subsection{$O(a * b)$}
Starting with the inner for-loop, it is run $b$ times in all cases. Then the outer for-loop is also run
$a$ times in every case. Since they are nested for loops, the complexities multiply. Therefore the overall
complexity is $O(a * b)$.

\section{Problem 4}
\begin{proof} In order to prove $(n + a)^{d} = \Theta{(n^{d})}$
we can prove $(n + a)^{d} = O(n^{d})$ and $(n + a)^{d} = \Omega{(n^{d})}$. first distribute the constant power.
\begin{equation*}
    (n + a)^{d} = n^{d} + dan + a^{d}
\end{equation*}
Now start with proving big O.
\begin{equation*}
    n^{d} + dan + a^{d} \le n^{d} + dan^d + a^{d}n^{d}
\end{equation*}
\begin{equation*}
    \le n^{d}(1 + da + a^{d})
\end{equation*}
Because $(1 + da + a^{d})$ is just a constant, this shows that $(n + a)^{d} = O(n^{d})$.
Now we can prove big Omega.
\begin{equation*}
    n^{d} + dan + a^{d} \ge n^{d} - dan^d - a^{d}n^{d}
\end{equation*}
\begin{equation*}
    \ge n^{d}(1 - da - a^{d})
\end{equation*}
Because $(1 - da - a^{d})$ is just a constant, this shows that $(n + a)^{d} = \Omega{(n^{d})}$.
Therefore, $(n + a)^{d} = \Theta{(n^{d})}$
\end{proof}






\end{document}
